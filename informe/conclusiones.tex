\section{Conclusiones}

Podemos concluir que el traceroute es una herramienta simple y poderosa a la hora de analizar la topología de internet y el funcionamiento de sus rutas. 

Pudimos notar que para rutas largas, la efectividad de detección de saltos intercontinentales parecería decaer, siempre tomando más saltos intercontinentales de los que realmente son. Esto es notorio en la ruta del primer experimento, que es el más lejano. Esto se puede deber a que las rutas de tal longitud podrían a ser poco estables en el tiempo, ya que las rutas pueden verse alteradas mientras se está haciendo el experimento. Sin embargo la versión modificada del método de Cimbala fue más efectiva, logrando una mejor performance a la hora de detectar saltos intercontinentales.

Sobre comportamientos anómalos, notamos que Missing Hops y False Round-Trip Times (nombrados en el paper de Jobsts) se vieron en todos los experimentos. El primero se debe a que muchos hosts deciden ignorar las respuestas ICMP. La anomalía denominada en el paper Missing Destination también se presentó, pero decidimos no incluir estos destinos en el experimento. No se hallaron situaciones anómalas con respecto a los links MPLS. Un experimento no incluído fue pegarle a la dirección de la web del MIT (\url{www.mit.edu}): cuando hicimos el análisis, esta tenía muy pocos saltos y la dirección final se encontraba localizada en Brasil. Después a través del servicio iplocation pudimos concluir que la IP era perteneciente a Akamai, una reconocida empresa de CDN, y por esto decidimos ignorar este experimento.
